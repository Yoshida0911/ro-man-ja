\documentclass[a4paper, 9pt]{styles/ozakilab_seminar}
\synctex=1

% ===== 基本パッケージ =====
\usepackage{graphicx}
\usepackage{amsmath,amssymb,bm}
\usepackage{tabularx,booktabs}
\usepackage{float}
\usepackage{stfloats}
\usepackage{url}
\usepackage{caption}
\usepackage{siunitx}
\usepackage{listings}
\usepackage{enumitem}
\usepackage{booktabs}
\usepackage{lmodern}
\usepackage{multirow}
\usepackage{subcaption}
\usepackage[dvipsnames]{xcolor}
\usepackage{tikz}

\newcommand{\legenddash}[1]{%
\tikz[baseline=-0.6ex] \draw[#1, dashed, line width=1pt] (0,0) -- (2em,0);
}

% \usepackage{subcaption}
% \captionsetup{font=small,labelfont=bf}
% \usepackage[hidelinks]{hyperref}

% ===== 文献 =====
% \usepackage[backend=biber,style=numeric,sorting=nty]{biblatex}
\usepackage[backend=biber,style=numeric-comp,sorting=none]{biblatex}
\addbibresource{refs.bib}

% ===== メイン =====
\begin{document}
 
% ===== メタ情報 =====
\title{日本語タイトル}
\etitle{English Title of the Paper}
\author{吉田 脩馬}
\eauthor{Shuma Yoshida}
\lab{尾崎・宮城・田畑 研究室}
% \keywords{}

% \maketitle
% ==== Abstract(EN) ====
\begin{abstract}
% \textbf{Abstract}\quad
Autonomous mobile robots face challenges in interpreting and integrating abstract contextual information, such as human states and social situations, into navigation. 
This paper proposes a novel framework that utilizes Vision-Language Model (VLM) to dynamically generate context-dependent parameters for Artificial Potential Field navigation.
Our approach leverages the VLM's advanced recognition capabilities to integrate multimodal inputs—specifically visual and acoustic Information—with the robot's assigned role to determine appropriate attraction/repulsion weights and intent flags. 
Experimental results from real-world trials on a university campus demonstrated that robots successfully exhibited adaptive behaviors, such as efficient avoidance or proactive approaches, enabling safer and socially compliant navigation.
\end{abstract}

% タイトル出力
\maketitle

% ==== 本文セクション====
% \section{諸言}
% 諸言を書きます.

% \section{関連研究}
% 関連研究を書きます.
% 参考文献は,文献\cite{akai}のように参照します.

% \section{提案手法}
% 提案手法を書きます.
% 数式は,\cref{gauss}のように参照します.
% \begin{equation}
%   P(x) = e
%   \label{gauss}
% \end{equation}

% \section{実験}
% 実験を書きます.
% 図は,\cref{fig:sample}のように参照します.
% \begin{figure}[H]
%   \centering
%   % figures/sample.png を置くか、パスを差し替え
%   \includegraphics[width=0.9\columnwidth]{figures/sample.png}
%   \caption{Figure caption (sample).}
%   \label{fig:sample}
% \end{figure}

% \section{考察}
% 考察を書きます.
% 表は,\cref{tab:sample}のように参照します.
% \begin{table}[H]
%   \centering
%   \caption{Table caption (sample).}
%   \label{tab:sample}
%   \begin{tabular}{lrr}
%     \toprule
%     Item & Value A & Value B \\
%     \midrule
%     Foo  & 1.23    & 4.56    \\
%     Bar  & 7.89    & 0.12    \\
%     \bottomrule
%   \end{tabular}
% \end{table}

% \section{結言}
% 結言を書きます.

% ==== 本文セクション ====
\input{part001}
\input{part002}
\input{part003}
\input{part004}
\input{part005}
\input{part006}

% ===== 参考文献 =====
\printbibliography

\end{document}
