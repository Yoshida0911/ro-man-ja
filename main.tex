\documentclass[a4paper, 9pt]{styles/ozakilab_seminar}
\synctex=1

% ===== 基本パッケージ =====
\usepackage{graphicx}
\usepackage{amsmath,amssymb,bm}
\usepackage{tabularx,booktabs}
\usepackage{float}
\usepackage{stfloats}
\usepackage{url}
\usepackage{caption}
\usepackage{siunitx}
\usepackage{listings}
\usepackage{enumitem}
\usepackage{booktabs}
\usepackage{lmodern}
\usepackage{multirow}

% \usepackage{subcaption}
% \captionsetup{font=small,labelfont=bf}
% \usepackage[hidelinks]{hyperref}

% ===== 文献 =====
% \usepackage[backend=biber,style=numeric,sorting=nty]{biblatex}
\usepackage[backend=biber,style=numeric-comp,sorting=none]{biblatex}
\addbibresource{refs.bib}

% ===== メイン =====
\begin{document}
 
% ===== メタ情報 =====
\title{日本語タイトル}
\etitle{English Title of the Paper}
\author{吉田 脩馬}
\eauthor{Shuma Yoshida}
\lab{尾崎・宮城・田畑 研究室}
% \keywords{}

% \maketitle
% ==== Abstract(EN) ====
\begin{abstract}
% \textbf{Abstract}\quad
This is a blank template for seminar papers using LuaLaTeX. 
\end{abstract}

% タイトル出力
\maketitle

% ==== 本文セクション====
% \section{諸言}
% 諸言を書きます.

% \section{関連研究}
% 関連研究を書きます.
% 参考文献は,文献\cite{akai}のように参照します.

% \section{提案手法}
% 提案手法を書きます.
% 数式は,\cref{gauss}のように参照します.
% \begin{equation}
%   P(x) = e
%   \label{gauss}
% \end{equation}

% \section{実験}
% 実験を書きます.
% 図は,\cref{fig:sample}のように参照します.
% \begin{figure}[H]
%   \centering
%   % figures/sample.png を置くか、パスを差し替え
%   \includegraphics[width=0.9\columnwidth]{figures/sample.png}
%   \caption{Figure caption (sample).}
%   \label{fig:sample}
% \end{figure}

% \section{考察}
% 考察を書きます.
% 表は,\cref{tab:sample}のように参照します.
% \begin{table}[H]
%   \centering
%   \caption{Table caption (sample).}
%   \label{tab:sample}
%   \begin{tabular}{lrr}
%     \toprule
%     Item & Value A & Value B \\
%     \midrule
%     Foo  & 1.23    & 4.56    \\
%     Bar  & 7.89    & 0.12    \\
%     \bottomrule
%   \end{tabular}
% \end{table}

% \section{結言}
% 結言を書きます.

% ==== 本文セクション ====
\input{part001}
\input{part002}
\input{part003}
\input{part004}
\input{part005}


% ===== 参考文献 =====
\printbibliography

\end{document}
